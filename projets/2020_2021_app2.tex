\documentclass{article}

\usepackage[T1]{fontenc}
\usepackage[utf8]{inputenc}
\usepackage[french]{babel}
\usepackage{amsmath, amssymb, amsfonts}
\usepackage{graphicx}
\usepackage{multicol}
\usepackage{tikz}
\RequirePackage[a4paper]{geometry}
\RequirePackage{fancyhdr}


\newcommand{\coursename}{AP1 - Algorithmique et programmation 1}
\fancyhead[L]
	{	\includegraphics	[height=1.1cm]
						{../../../../latex/UPEM-IGM-V1_300dpi.png}
	}
\fancyhead[C]
	{	{\bf \coursename}\\
		L1 Informatique \& L1 Mathématiques\\
		Semestre 1
	}
\setlength{\headheight}{50pt}
\renewcommand{\headrulewidth}{1pt}


\title{Approche par problème 2\\ Bet \& Risk, le jeu}
\author{}
\date{}


\begin{document}

\maketitle
\thispagestyle{fancy}

Le code source du programme de jeu de paris {\tt Bet \& Risk} a été perdu à cause d'une erreur de manipulation.

\section{Bet \& Risk}

Les ingénieurs ont aussi perdu la spécification du langage, mais heureusement, ils ont retrouvé un exemple
d'une suite d'instructions et son résultat :

\setlength{\columnseprule}{0.4pt}
\setlength{\marginparwidth}{0pt}

\begin{multicols}{2}
\begin{verbatim}
<<< graine
42
<<< lancer
>>> Vous avez tiré un 6
>>> La banque a tiré un 1
>>> Votre mise ?
2
>>> Vous avez maintenant tiré un 1
>>> La banque a aussi tiré un 6 et un 3
Votre somme : 7
Celle de la banque : 9
>>> Perdu !
>>> Quitte ou double ?
quitte
<<< consulter
>>> Vous avez 8€
<<< lancer
>>> Vous avez tiré un 2
>>> La banque a tiré un 2
>>> Votre mise ?
0
>>> Pas possible !
>>> Pour info, vous avez 8€
>>> Votre mise ?
1
>>> Vous avez maintenant tiré un 2
>>> La banque a aussi tiré un 6 et un 1
Votre somme : 4
Celle de la banque : 8
>>> Perdu !
>>> Quitte ou double ?
double
>>> Vous avez maintenant tiré un 6
>>> La banque a aussi tiré un 6
Votre somme : 8
Celle de la banque : 12
>>> Perdu !
>>> Quitte ou double ?
quitte
<<< lancer
>>> Vous avez tiré un 5
>>> La banque a tiré un 1
>>> Votre mise ?
10
>>> Pas possible !
>>> Pour info, vous avez 6€
>>> Votre mise ?
6
>>> Vous avez maintenant tiré un 5
>>> La banque a aussi tiré un 4 et un 1
Votre somme : 10
Celle de la banque : 5
>>> Gagné !
<<< lancer
>>> Vous avez tiré un 1
>>> La banque a tiré un 1
>>> Votre mise ?
3
>>> Vous avez maintenant tiré un 2
>>> La banque a aussi tiré un 2 et un 5
Votre somme : 3
Celle de la banque : 7
>>> Perdu !
>>> Quitte ou double ?
double
>>> Vous avez maintenant tiré un 5
>>> La banque a aussi tiré un 1
Votre somme : 7
Celle de la banque : 7
>>> Egalité !
>>> Quitte ou double ?
double
>>> Vous avez maintenant tiré un 5
>>> La banque a aussi tiré un 2
Votre somme : 10
Celle de la banque : 7
>>> Gagné !
<<< consulter
>>> Vous avez 24€
<<< terminer
Bravo, vous avez doublé votre mise initiale !
\end{verbatim}
\end{multicols}

Malgré tout, les ingénieurs ont aussi quelques souvenirs du fonctionnement du programme :

\paragraph{Le fonctionnement du jeu.} Tout d'abord, le joueur et la banque lance chacun un dé. Ensuite, le joueur décide de sa mise. Enfin, le joueur relance un dé alors que la banque en relance deux ! Gagne le pari celui qui a la meilleure somme de points sur les deux meilleurs dés (voir par exemple la figure \ref{exemple}). Si le joueur perd, il peut tenter de conjurer le sort grâce à un système de \emph{Quitte ou double}, si l'argent qu'il a le lui permet.

\begin{figure}[h]
	\begin{tikzpicture}
		\node (cinq rouge) at (0,4) {\includegraphics[scale=0.375]{fig/Cinq_rouge}};
		\node (un blanc) at (5.5,4) {\includegraphics[scale=0.4875]{fig/Un_blanc}};
		\node (deux rouge) at (0,0) {\includegraphics[scale=0.375]{fig/Deux_rouge}};
		\node (six blanc) at (4,0) {\includegraphics[scale=0.4875]{fig/Six_blanc}};
		\node (trois blanc) at (7, 0) {\includegraphics[scale=0.4875]{fig/Trois_blanc}};

		\draw (-5, 4.5)	node	{\textsc{Premier lancer :}};
		\draw (-5, 0.5)	node	{\textsc{Second lancer :}};	
		
		\draw (0,2.25) ellipse (2cm and 3.5cm);
		\draw (5.5,0.05) ellipse (3cm and 1.9cm);

		\draw (0, -2.05)	node	{\textsc{Somme = 7}};	
		\draw (5.5, -2.05)	node	{\textsc{Somme = 9}};	
	\end{tikzpicture}
	\caption	{	\textsc{Déroulement d'un pari.}
						\\
						Au premier lancé, le joueur peut être tenté de miser gros.
						\\
						Néanmoins, sa somme finale n'est que de $7$, alors que celle de la banque vaut $9$ :
						il perd alors sa mise
					}
		\label{exemple}
\end{figure}

\paragraph{But du jeu.} Un ingénieur se souvient que l'objectif du jeu était d'enchainer les paris pour
au moins doubler sa mise initiale. Un autre ingénieur se souvient qu'il n'y avait pas de message de fin de jeu
lorsque cet objectif n'était pas atteint, par contre s'il était atteint, un message de félicitations était affiché.

\paragraph{Le langage utilisé pour jouer.} Les seules entrées autorisées au clavier sont les mots
{\tt graine}, {\tt lancer}, {\tt consulter}, {\tt quitte}, {\tt double} et {\tt terminer} :
\begin{center}
	\begin{tabular}{ll}
		{\tt graine}			&	Fxer la graine (\emph{seed} en anglais) responsable des valeurs renvoyé
		\\
								&	par les générateurs de nombres aléatoires ;
		\\
		{\tt lancer}		&	Lance le début des ostilités : la manche commence en lancant les
		\\
								&	premiers dés ;
		\\
		{\tt consulter}	&	Vérifier l'argent restant au joueur ;	
		\\
		{\tt quitte}			&	Met fin à un pari ;
		\\
		{\tt double}		&	Double la mise du pari en cours, si la mise reste inférieure à l'argent du joueur;
		\\
		{\tt terminer}		&	Termine le jeu.
	\end{tabular}
\end{center}


\paragraph{Le prompt.} Le prompt d'entrée est {\tt <}{\tt <}{\tt <}, alors que le prompt de sortie est {\tt >}{\tt >}{\tt >}.
A aucun moment, le prompt d'entrée n'est entré par l'utilisateur.


\bigskip
\null
\bigskip
\null
\bigskip


\noindent
\textbf{Votre défi :}

\bigskip

\begin{center}
	\begin{minipage}{10cm}
		\textsc	{Reécrire le code source en Python du jeu {\tt Bet \& Risk} de manière à respecter
					le rendu retrouvé et les souvenirs des ingénieurs.}
	\end{minipage}
\end{center}

\bigskip
\null
\bigskip





\newpage

\section{Astuces de développement}

\subsection{Fixer la graine}

La commande {\tt graine} disponible dans le langage est en fait une \emph{commande cachée},
c'est-à-dire non divulguée aux joueurs. Elle est uniquement implémenté pour pouvoir reproduire
le comportement des lancers de dés.

\medskip

Par exemple, le programme suivant permet de fixer la graine à $42$ et d'afficher la valeurs
des cent premiers lancers de dés (un par ligne)

\begin{verbatim}
from random import seed
from random import randint

for i in range(100):
    print(randint(1, 6))
\end{verbatim}
Le résultat de son exécution, une fois ramené sur plusieurs lignes est :
\begin{center}
	\begin{verbatim}
	6 1 1 6 3 2 2 2 6 1 6 6 5 1 5 4 1 1 1 2 2 5 5 1 5 2 6 6 6
	5 4 2 4 5 3 1 2 6 4 3 3 2 2 3 1 1 4 1 3 3 5 3 1 6 4 5 1 4
	1 5 3 6 5 3 5 2 6 1 1 6 2 3 1 2 1 4 3 4 6 3 2 3 3 2 6 3 6
	6 6 1 5 6 2 5 6 2 2 4 4 3
	\end{verbatim}
\end{center}

\subsection{Les redirections}

Lors du développement, les ingénieurs ont utilisé un système Unix (e.g., Linux ou macOS)
pour ne pas être obligés de ressaisir toutes les instructions au clavier à chaque fois
qu'ils relancaient leur programme.

En effet, il suffit de stocker dans un fichier les commandes qui seraient tapé dans le terminal,
puis d'utiliser les redirections d'entrées/sorties du shell.


Comment cela fonctionne t-il concrètement ? Supposons, par exemple,
que vous avez écrit le programme suivant dans un fichier \texttt{test.py}:
\begin{verbatim}
    s = input()
    print(s)
\end{verbatim}
Alors, votre première option est de simplement l'exécuter avec la commande:
\begin{verbatim}
    python3 test.py
\end{verbatim}
Si ensuite, vous saisissez \texttt{Bonjour} au clavier,
cette chaîne de caractères s'affichera à l'écran.
Mais alternativement,
vous pouvez aussi créer un fichier \texttt{message.txt}
qui contient seulement le mot \texttt{Bonjour},
et ensuite exécuter votre programme avec la commande:
\begin{verbatim}
    python3 test.py < message.txt
\end{verbatim}
Vous ne pourrez alors rien saisir au clavier,
mais le programme affichera toujours \texttt{Bonjour}.
On peut aussi renvoyer la sortie sur un autre fichier :
\begin{verbatim}
    python3 test.py < message.txt > sortie.txt
\end{verbatim}
Cette fois, le programme ne permet pas de servir le moindre texte
et n'écrit rien. Par contre, le fichier {\tt sortie.txt} a été crée : il contient
le texte \texttt{Bonjour}.

\bigskip

Dans le cas de l'exemple retrouvé par les ingénieurs, le fichier d'entrée et de sortie sont :

\begin{multicols}{2}
\begin{verbatim}
graine
42
lancer
2
quitte
consulter
lancer
0
1
double
quitte
lancer
10
6
lancer
3
double
double
consulter
terminer

\end{verbatim}

\begin{center}
	\phantom{Bravo, vous avez doublé votre mise initiale !}
	\textit{Le fichier d'entrée}
	\phantom{Bravo, vous avez doublé votre mise initiale !}
	\phantom{Bravo, vous avez doublé votre mise initiale !}
	\phantom{Bravo, vous avez doublé votre mise initiale !}
	\phantom{Bravo, vous avez doublé votre mise initiale !}
\end{center}


\begin{verbatim}
>>> Vous avez tiré un 6
>>> La banque a tiré un 1
>>> Votre mise ?
>>> Vous avez maintenant tiré un 1
>>> La banque a aussi tiré un 6 et un 3
Votre somme : 7
Celle de la banque : 9
>>> Perdu !
>>> Quitte ou double ?
>>> Vous avez 8€
>>> Vous avez tiré un 2
>>> La banque a tiré un 2
>>> Votre mise ?
>>> Pas possible !
>>> Pour info, vous avez 8€
>>> Votre mise ?
>>> Vous avez maintenant tiré un 2
>>> La banque a aussi tiré un 6 et un 1
Votre somme : 4
Celle de la banque : 8
...
>>> Gagné !
>>> Vous avez 24€
Bravo, vous avez doublé votre mise initiale !
\end{verbatim}

\begin{center}
	\phantom{Bravo, vous avez doublé votre mise initiale !}
	\textit{Le fichier obtenu en sortie}
	\phantom{Bravo, vous avez doublé votre mise initiale !}
\end{center}
\end{multicols}

\newpage

\section{Connaissances nécessaires à la réalisation de l'APP 2 }

Les connaissances suivantes sont suffisantes pour réécrire le code source du jeu Bet \& Risk :

\bigskip

\begin{itemize}
	\item[$\bullet$]	Notions de valeurs, types et variables ;
	\item[]
	\item[$\bullet$] Les conditionnelles ;
	\item[]
	\item[$\bullet$] Les boucles.
\end{itemize}

\bigskip
\null
\bigskip

\begin{center}
	$\star$ \hspace{2cm} $\star$
	
	\bigskip
	\null
	\bigskip

	$\star$
\end{center}

\end{document}

