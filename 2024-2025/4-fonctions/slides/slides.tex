\documentclass[x11names,svgnames]{beamer}

\usetheme{metropolis}
\setbeamertemplate{caption}{\raggedright\insertcaption\par}
\setbeamercolor{block title}{fg=LightSkyBlue4}
\newenvironment<>{importantblock}[1]{%
  \begin{actionenv}#2%
    \def\insertblocktitle{#1}%
    \par%
    \mode<presentation>{%
      \setbeamercolor{block title}{%
        use=normal text,
        fg=normal text.fg,
        bg=normal text.bg!80!fg
      }
      \setbeamercolor{block body}{
        use={block title, normal text},
        bg=block title.bg!50!normal text.bg
      }
    }%
    \usebeamertemplate{block begin}}
  {\par\usebeamertemplate{block end}\end{actionenv}}
\newenvironment<>{decisionblock}[1][Statement of the problem]{%
  \begin{actionenv}#2%
    \def\insertblocktitle{#1}%
    \par%
    \mode<presentation>{%
      \setbeamercolor{block title}{%
        use=normal text,
        fg=normal text.fg,
        bg=normal text.bg!80!fg
      }
      \setbeamercolor{block body}{
        use={block title, normal text},
        bg=block title.bg!50!normal text.bg
      }
    }%
    \usebeamertemplate{block begin}}
  {\par\usebeamertemplate{block end}\end{actionenv}}
\usepackage{fontspec}
\usepackage[french, frenchkw]{algorithm2e}
\SetKwFor{Pc}{Pour chaque}{faire}{fin}
\SetKwProg{Fn}{Fonction}{:}{}
\usepackage{textcomp}
\usepackage{minted}
\usepackage{pdfpages}
\usepackage{qrcode}
\usepackage[francais]{babel}
\usepackage{xspace}
\usepackage[normalem]{ulem}
\usepackage[backend=biber,style=authoryear,url=false]{biblatex}
\addbibresource{Bibliography.bib} % Bibliography file
\usepackage{amsmath}
\usepackage{mathtools}
\usepackage{pifont}% http://ctan.org/pkg/pifont
\newcommand{\cmark}{\ding{51}}%
\newcommand{\xmark}{\ding{55}}%
\usepackage{marvosym}
\usepackage{tikz}
\usetikzlibrary{calc, positioning, arrows, automata, fadings, decorations.pathmorphing, shapes, patterns, tikzmark, fit, chains, snakes, shapes.multipart}
% https://tex.stackexchange.com/questions/401884/how-do-i-change-hyperlinks-color-only
\hypersetup{
  colorlinks,
  allcolors=.,
  urlcolor=DarkBlue,
}


\author{}
\date{Lundi 2 octobre 2023}

\title{\includegraphics[width=0.75\textwidth]{../../img/logo-igm.png} \\ Algorithmique et Programmation 1}
\institute{L1 Mathématiques - L1 Informatique \\ Semestre 1}

\begin{document}

\maketitle

\begin{frame}{Retour sur le CC0}
  \begin{block}{Notes}
    \vspace{0pt}
    Dans la semaine, probablement d'ici mercredi
  \end{block}
  \begin{block}{Questions d'algorithmique}
    \begin{itemize}
    \item Plus difficiles\dots \pause
    \item \dots surtout avec une erreur dans une question !
    \item Nous allons fixer les termes
    \end{itemize}
  \end{block}
  \pause
  \begin{block}{Sur la triche}
    \begin{itemize}
    \item L'examen est \textbf{individuel}
    \item Tolérance zéro aux CC 1, 2 et 3
    \end{itemize}
  \end{block}
\end{frame}

\section{Chapitre 4 : Fonctions (suite)}

\begin{frame}
  \begin{block}{Fonction}
    \vspace{0pt}
    En informatique, une fonction est :
    \begin{itemize}
    \item Un morceau de programme
    \item Portant en général \emph{un nom}
    \item Prenant un ou plusieurs \emph{paramètres} (ou zéro)
    \item Renvoyant un résultat (la plupart du temps)
    \end{itemize}
  \end{block}
\end{frame}

\section{Exemples}

\begin{frame}[fragile]{Fonction à paramètres et résultats}
  Calculer le \emph{minimum} de deux entiers (dans Thonny) :
  \pause
\begin{minted}{python}
def maximum(a,b):
    if a <= b:
        return a
    else:
        return b
\end{minted}
\end{frame}

\begin{frame}[fragile]{Fonction sans valeur de retour}
  \begin{columns}
    \begin{column}{0.7\textwidth}
  Dessiner un carré fait du caractère \mintinline{python}{caractere} (dans Thonny) :
\end{column}
\begin{column}{0.1\textwidth}
\begin{verbatim}
*****
*****
*****
*****
*****
\end{verbatim}
\end{column}
\end{columns}
\pause
\vspace{-0.5em}
\begin{minted}{python}
def dessine_carre(n, caractere):
    i = 0
    while i < n:
        j = 0
        while j < n:
            print(caractere, end = '')
            j += 1
        print('\n', end = '')
        i += 1
    return
\end{minted}
\end{frame}

\begin{frame}
  \frametitle{Composition de fonctions}

  \begin{block}{En maths}
    $$\forall x \in X, g \circ f(x) = g(f(x))$$
  \end{block}
  
  \begin{block}{En informatique}
    \vspace{0pt}
    On peut appeler une fonction dans une fonction ! \newline (et ainsi de suite)
  \end{block}
\end{frame}

\begin{frame}[fragile]{Dessiner un carré : variante}
  \begin{columns}
    \begin{column}{0.1\textwidth}
\begin{verbatim}
*****
*****
*****
*****
*****
\end{verbatim}
    \end{column}
    \begin{column}{0.7\textwidth}
      Carré de côté \mintinline{python}{n} = \mintinline{python}{n} lignes de longueur \mintinline{python}{n}
    \end{column}
\end{columns}
\bigskip
On peut donc décomposer ainsi :
  \begin{itemize}
  \item \mintinline{python}{dessine_ligne(n,car)} :\newline dessine \emph{une ligne} de longueur \mintinline{python}{n} composée du caractère \mintinline{python}{car}
  \item \mintinline{python}{dessine_carre(n,car)} :\newline dessine \emph{un carré} de longueur \mintinline{python}{n} composé du caractère \mintinline{python}{car}
  \end{itemize}
  \MVRightarrow{} Ce que ça donne dans Thonny
\end{frame}
\begin{frame}[fragile]{Dessiner un carré : variante}
\begin{minted}{python}
def dessine_ligne(n, caractere):
 """dessine une ligne de longueur n
    composée du caractère car"""
    j = 0
    while j < n:
        print(caractere, end = '')
        j += 1
    print('\n', end = '')
    return
\end{minted}
\end{frame}
\begin{frame}[fragile]{Dessiner un carré : variante}
\begin{minted}{python}
def dessine_carre(n, caractere):
 """dessine un carré de longueur n
    composé du caractère car"""
    i = 0
    while i < n:
        dessine_ligne(n,caractere)
        i += 1
    return
\end{minted}
\end{frame}

\begin{frame}[fragile]{Erreur fréquente : confusion \begin{tabular}{l}paramètre / saisie \\ retour / affichage\end{tabular}}

  \begin{block}{Paramètre / saisie}
\begin{minted}{python}
def minimum(a,b):
    a = int(input()) # NON !
    b = int(input()) # NON !
    if a <= b:
        return a
    else:
        return b
\end{minted}
  \end{block}

  \begin{block}{Retour / affichage}
\begin{minted}{python}
def maximum(a,b):
    if a >= b:
        print(a) # NON !
    else:
        print(b) # NON !
\end{minted}
  \end{block}
\end{frame}

\begin{frame}
  \frametitle{Fonctions et espaces de nom}
  \begin{itemize}
  \item Les paramètres et variables définies dans le corps d'une fonction sont indépendantes des autres variables du programme
  \item Elles n'existent plus une fois l'exécution de la fonction terminée
  \item On les appelle des variables \emph{locales}
\end{itemize}

Par conséquent, on peut renommer les variables d'une fonction.

$\rightarrow{}$ Démonstration sur Thonny (\texttt{minimum.py}).

\end{frame}

\begin{frame}
  \frametitle{(Contre-)exemple : intervertir des variables}
  Dans Thonny : \texttt{echange.py}
  \pause
  \begin{itemize}
  \item changer les valeurs de \mintinline{python}{a} et \mintinline{python}{b} dans la fonction \emph{n'a pas d'effet} sur `x` et `y` dans le programme principal !
  \item la variable \mintinline{python}{temp} n'existe plus après l'exécution de la fonction
  \end{itemize}
\end{frame}

\begin{frame}
  \frametitle{Sous le capot : sémantique d'un appel}

  \begin{block}{Espace de noms}
    \vspace{0pt}
    Ensemble de noms (variables, fonctions) défini à un certain point d'un programme
  \end{block}

  \MVRightarrow{} lors d'un appel de fonction, création d'un espace de nom \emph{local} :
  \begin{itemize}
  \item Paramètres $\rightarrow$ leur valeur lors de l'appel
  \item Variables locales
  \end{itemize}

  \begin{block}{Exemple}
    \vspace{0pt}
    \texttt{minimum\_2.py} dans \href{https://pythontutor.com/}{Python Tutor} et Thonny.
  \end{block}

\end{frame}

\begin{frame}
  \frametitle{Espaces de noms lors de l'exécution}
  À un point du programme, l'ensemble des noms (variables, fonctions) connus est constitué de plusieurs espaces imbriqués (du plus ancien au plus récent) :
  \begin{itemize}
  \item espace de noms prédéfini (\mintinline{python}{print}, \mintinline{python}{input}, \dots)
  \item espace de noms global : définis dans le programme principal
  \item espaces de noms locaux imbriqués dans l'\emph{ordre chronologique} des appels
    \begin{itemize}
    \item paramètres de l'appel
    \item variables locales à la fonction
    \end{itemize}
  \end{itemize}
\end{frame}

\begin{frame}[fragile]
  \frametitle{Pile d'appels}
  L'empilement des espaces de noms obéit à une politique de \emph{pile}

  \begin{itemize}
  \item sommet : appel en cours
  \item en-dessous : appels précédents
  \item (presque) tout en bas : espace de nom global
  \end{itemize}

  \begin{columns}
    \begin{column}{0.5\textwidth}
\begin{minted}{python}
def f(x):
    return g(x) + 1
def g(x):
    return h(x) + 2
def h(x):
    return x + 3
\end{minted}
    \end{column}
    \begin{column}{0.4\textwidth}
      \alt<1-4>{%
      \begin{tikzpicture}
        % Stack rectangle
        \draw (0,0) rectangle (2,3);
        % Stack top
        \node at (1,3.3) {Sommet};
        % Stack elements
        \pause
        \draw (0.2,0.5) rectangle (1.8,0.8) node[midway] {f};
        \pause
        \draw (0.2,0.9) rectangle (1.8,1.2) node[midway] {g};
        \pause
        \draw (0.2,1.3) rectangle (1.8,1.6) node[midway] {h};
        % Stack direction
        \draw[->,ultra thick] (2.5,0) -- (2.5,3);
      \end{tikzpicture}%
    }{%
      \begin{tikzpicture}[yscale=-1]
        % Stack rectangle
        \draw (0,0) rectangle (2,3);
        % Stack top
        \node at (1,3.3) {Sommet};
        % Stack elements
        \pause
        \draw (0.2,0.5) rectangle (1.8,0.8) node[midway] {f};
        \pause
        \draw (0.2,0.9) rectangle (1.8,1.2) node[midway] {g};
        \pause
        \draw (0.2,1.3) rectangle (1.8,1.6) node[midway] {h};
        % Stack direction
        \draw[->,ultra thick] (2.5,0) -- (2.5,3);
      \end{tikzpicture}%
    }
    \end{column}
  \end{columns}

  \medskip

  \pause

  \textbf{Attention !} Dans Python Tutor, le plus récent est en bas.
\end{frame}

\begin{frame}
  \frametitle{Pile d'appels}

  \begin{block}{Quand un nouvel appel commence}
    \begin{itemize}
    \item l'exécution de la fonction en cours s'interrompt
    \item un nouvel espace de noms local est créé
    \item l'exécution de la fonction appelée commence
    \end{itemize}
  \end{block}


  \begin{block}{Quand l'appel en cours se termine}
    \begin{itemize}
    \item son espace de nom est supprimé de la pile
    \item l'exécution de l'appel précédent reprend
    \end{itemize}

  \end{block}

\end{frame}
\begin{frame}{Portée des variables}

  \begin{block}{Accès à la valeur d'une variable}
  \begin{itemize}
  \item possible pour n'importe quel nom défini dans un des espaces de noms antérieurs
  \item si plusieurs espaces contiennent le même nom, c'est le plus récent qui est sélectionné
  \end{itemize}
\end{block}

\begin{block}{Affectation}
  
  \begin{itemize}
  \item par défaut, uniquement aux variables locales
  \item pour une variable dans un espace de nom plus ancien, mots-clés \mintinline{python}{global} ou \mintinline{python}{nonlocal} (à utiliser avec précaution)
  \end{itemize}

\end{block}

\end{frame}

\begin{frame}[fragile]{Déroulement détaillé d'un appel}


  \begin{columns}[t]
    \begin{column}{0.3\textwidth}
      Considérons l'appel :
\begin{minted}{python}
def f(p_1, ..., p_n):
...
return expr

# Appel de fonction
# (à l'intérieur
#  d'une expression)
... f(e_1, ..., e_n)
\end{minted}
    \end{column}

    \begin{column}{0.6\textwidth}
      Succession des étapes de l'appel :
      \begin{itemize}
        \alt<1>{%
      \item création d'un espace de noms local contenant \mintinline{python}{p_1} à \mintinline{python}{p_n} au sommet de la pile d'appels
      \item chaque expression \mintinline{python}{e_i} est évaluée en une valeur \mintinline{python}{v_i} et affectée à la variable \mintinline{python}{p_i}
      \item exécution du corps de la fonction dans l'espace de noms local%
      }{%
      \item si la fonction exécute l'instruction \mintinline{python}{return expr} ou atteint la fin de son bloc d'instructions
      \item l'espace de noms local est détruit
      \item l'expression appelante \mintinline{python}{f(e_1, ..., e_n)} prend la valeur de \mintinline{python}{expr} (respectivement \mintinline{python}{None})
       \item reprise du programme principal dans l'espace global
         }
      \end{itemize}

    \end{column}
  \end{columns}

  \pause

  \begin{itemize}
  \item les noms \mintinline{python}{p_1} à \mintinline{python}{p_n} sont appelés paramètres (formels)
  \item les valeurs \mintinline{python}{v_1} à \mintinline{python}{v_n} sont appelées paramètres effectifs \newline (ou arguments)
    \end{itemize}


\end{frame}
\begin{frame}[fragile]{Documentation et test de fonctions}
  \begin{block}{Chaînes de documentation (\mintinline{python}{docstring})}
    \vspace{0pt}

    Bonne pratique : indiquer par un commentaire
    \begin{itemize}
    \item à quoi sert une fonction
     \item ce que représentent ses paramètres et leur type
     \item ce que représente sa valeur de retour
     \item d'éventuels effets ou causes secondaires
    \end{itemize}
  \end{block}

\begin{minted}{python}
def triple(n):
  """
  Fonction calculant le triple du nombre n (int ou float)
  ou la répétition trois fois de la chaîne n.
  """
  return n * 3
\end{minted}
\end{frame}
\begin{frame}[fragile]{Documentation et test de fonctions}
  \begin{itemize}
    \item On peut accéder à la chaîne de documentation d'une fonction en tapant \mintinline{python}{help(nom de la fonction)} dans l'interpréteur
  \item Cela fonctionne aussi pour les fonctions prédéfinies ou issues de modules
  \end{itemize}
\end{frame}
\begin{frame}[fragile]{Tests intégrés à la documentation (\mintinline{python}{doctest})}
  \begin{itemize}
  \item Toute fonction \emph{doit être testée immédiatement} pour s'assurer qu'elle fonctionne.
  \item On peut intégrer les tests à la documentation de la fonction.
  \end{itemize}
\begin{minted}{python}
def triple(n):
  """
  Fonction calculant le triple du nombre n (int ou float)
  ou la répétition trois fois de la chaîne n.
  
  >>> triple(3)
  9
  >>> triple(9.0)
  27.0
  """
  return n * 3
\end{minted}
\end{frame}
\begin{frame}[fragile,fragile]{Tests intégrés à la documentation (\mintinline{python}{doctest})}
  Il existe des outils qui permettent de lancer automatiquement tous les tests présents dans la documentation, et de vérifier qu'ils produisent les résultats annoncés.

  Par exemple, à la fin d'un programme, on peut écrire le code suivant pour lancer systématiquement tous les tests présents dans le fichier :
\begin{minted}{python}
  import doctest
  doctest.testmod()
\end{minted}
\end{frame}

\end{document}

%%% Local Variables:
%%% TeX-engine: xetex
%%% TeX-command-extra-options: "-shell-escape"
%%% End: