\documentclass[x11names,svgnames]{beamer}

\usetheme{metropolis}
\setbeamertemplate{caption}{\raggedright\insertcaption\par}
\setbeamercolor{block title}{fg=LightSkyBlue4}
\newenvironment<>{importantblock}[1]{%
  \begin{actionenv}#2%
    \def\insertblocktitle{#1}%
    \par%
    \mode<presentation>{%
      \setbeamercolor{block title}{%
        use=normal text,
        fg=normal text.fg,
        bg=normal text.bg!80!fg
      }
      \setbeamercolor{block body}{
        use={block title, normal text},
        bg=block title.bg!50!normal text.bg
      }
    }%
    \usebeamertemplate{block begin}}
  {\par\usebeamertemplate{block end}\end{actionenv}}
\newenvironment<>{decisionblock}[1][Statement of the problem]{%
  \begin{actionenv}#2%
    \def\insertblocktitle{#1}%
    \par%
    \mode<presentation>{%
      \setbeamercolor{block title}{%
        use=normal text,
        fg=normal text.fg,
        bg=normal text.bg!80!fg
      }
      \setbeamercolor{block body}{
        use={block title, normal text},
        bg=block title.bg!50!normal text.bg
      }
    }%
    \usebeamertemplate{block begin}}
  {\par\usebeamertemplate{block end}\end{actionenv}}
\usepackage{fontspec}
\usepackage[french, frenchkw]{algorithm2e}
\SetKwFor{Pc}{Pour chaque}{faire}{fin}
\SetKwProg{Fn}{Fonction}{:}{}
\usepackage{textcomp}
\usepackage{minted}
\usepackage{pdfpages}
\usepackage{qrcode}
\usepackage[francais]{babel}
\usepackage{xspace}
\usepackage[normalem]{ulem}
\usepackage[backend=biber,style=authoryear,url=false]{biblatex}
\addbibresource{Bibliography.bib} % Bibliography file
\usepackage{amsmath}
\usepackage{mathtools}
\usepackage{pifont}% http://ctan.org/pkg/pifont
\newcommand{\cmark}{\ding{51}}%
\newcommand{\xmark}{\ding{55}}%
\usepackage{marvosym}
\usepackage{tikz}
\usetikzlibrary{calc, positioning, arrows, automata, fadings, decorations.pathmorphing, shapes, patterns, tikzmark, fit, chains, snakes, shapes.multipart}
% https://tex.stackexchange.com/questions/401884/how-do-i-change-hyperlinks-color-only
\hypersetup{
  colorlinks,
  allcolors=.,
  urlcolor=DarkBlue,
}


\author{}
\date{Lundi 23 septembre 2024}

\title{\texorpdfstring{\includegraphics[width=0.75\textwidth]{../../img/logo-igm.png} \\[1.5em] Algorithmique et Programmation 1}{Algorithmique et Programmation 1}}
\institute{L1 Mathématiques - L1 Informatique \\ Semestre 1}

\begin{document}

\maketitle

\section{Chapitre 4 : Fonctions}

\begin{frame}[fragile]{Pas facile à lire !}

\begin{minted}{python}
i = 0
while i < n:
    j = 0
    while j < n:
        print(caractere, end = '')
        j += 1
    print('\n', end = '')
    i += 1
\end{minted}
\end{frame}

\begin{frame}{Une fonction : qu'est-ce que c'est ?}
    En informatique, une fonction est :
    \begin{itemize}
    \item Un morceau de programme
    \item Portant en général \emph{un nom}
    \item Prenant un ou plusieurs \emph{paramètres} (ou zéro)
    \item Renvoyant un résultat (la plupart du temps)
    \item[\MVRightarrow{}] Ça ressemble beaucoup aux fonctions vues en mathématiques !
    \end{itemize}
\end{frame}

\begin{frame}
  \frametitle{Une fonction : à quoi ça sert ?}
  \vspace{1em}
  \centering
  \includegraphics[width=0.8\textwidth]{img/spaghetti.png}
  \end{frame}

\begin{frame}
  \frametitle{Une fonction : à quoi ça sert ?}

  \begin{block}{Lisibilité}
    \begin{itemize}
    \item Isoler les parties du programme selon ce qu'elles font
    \item Éviter trop d'imbrications entre \mintinline{python}{if} et \mintinline{python}{while}
    \end{itemize}
  \end{block}

  \pause

  \begin{block}{Modularité et robustesse}
    \begin{itemize}
    \item Réutiliser le code plusieurs fois
    \item Faciliter la correction des bugs
    \end{itemize}
  \end{block}

  \pause
  \begin{block}{Généricité}
    \begin{itemize}
    \item Changer la valeur des paramètres
    \end{itemize}
  \end{block}
\end{frame}

\begin{frame}{Fonctions prédéfinies}
  \MVRightarrow{} On en a déjà vu !

  \begin{exampleblock}{Exemples}
    \begin{itemize}
    \pause
  \item \mintinline{python}{print(s)} : afficher la chaîne de caractères \mintinline{python}{s}
  \item \mintinline{python}{input(s)} : afficher la chaîne de caractères \mintinline{python}{s} et attendre une entrée utilisateur
  \item \mintinline{python}{int(x)} : convertit l'objet \mintinline{python}{x} en entier
  \item \mintinline{python}{len(x)} : renvoie la longueur de l'objet \mintinline{python}{x}
  \end{itemize}
\end{exampleblock}

\pause

\begin{block}{Modules}
  \vspace{0pt}
  Bibliothèque de fonctions, de types et d'objets
\end{block}

\begin{exampleblock}{Exemples}
  \pause
  \begin{itemize}
  \item \mintinline{python}{randint(a,b)} (module \mintinline{python}{randint})
  \item[\MVRightarrow{}] renvoie un entier aléatoire compris entre \mintinline{python}{a} et \mintinline{python}{b}
  \end{itemize}
  \end{exampleblock}
  
\end{frame}

\begin{frame}[fragile]
  \frametitle{Définition et appel de fonction}

  \begin{block}{Définir une fonction}
    \vspace{0pt}
    La syntaxe pour définir une fonction est :
    
\begin{minted}{python}
def nom_fonction(parametre_1, ..., parametre_n):
    # corps de la fonction
    # utilisant parametre_1, ..., parametre_n
    ...
    # peut renvoyer un résultat :
    return resultat
\end{minted}
  \end{block}
  \pause
  \begin{block}{Appeler une fonction}
    \vspace{0pt}
    On peut ensuite \emph{appeler} la fonction \mintinline{python}{nom_fonction} dans le code :
\begin{minted}{python}
resultat = nom_fonction(expression_1, ..., expression_n)
\end{minted}
  \end{block}
\end{frame}

\section{Exemples}

\begin{frame}[fragile]{Fonction à paramètres et résultats}
  Calculer le maximum de deux entiers :
  \pause
\begin{minted}{python}
def maximum(a,b):
    if a >= b:
        return a
    else:
        return b
\end{minted}
\pause
  \begin{block}{Entraînement}
    \begin{itemize}
    \item Ouvrir le programme \texttt{maximum.py}
    \item Décrire l'exécution pas à pas du programme (avec état de la mémoire)
    \item On peut aussi essayer avec \href{https://pythontutor.com/}{Python Tutor}
    \item Dresser un tableau de valeurs de l'exécution du programme
    \end{itemize}
  \end{block}
\end{frame}

\begin{frame}[fragile]{Fonction sans paramètres}
  Simuler un lancé de dés, et compter combien de coups il faut pour faire un 6.
  \pause
\begin{minted}{python}
from random import randint
def lance_de():
    return randint(1,6)

compteur = 1
while lance_de() != 6:
    compteur = compteur + 1
print('Obtenu un 6 en', compteur, 'jets de dé.')
\end{minted}
  \pause
  Ce que ça donne sur \href{https://pythontutor.com/}{Python Tutor}.
\end{frame}

\begin{frame}[fragile]{Fonction sans valeur de retour}
  Dessiner un carré fait du caractère \mintinline{python}{caractere} :
  \pause
\begin{minted}{python}
def dessine_carre(n, caractere):
    i = 0
    while i < n:
        j = 0
        while j < n:
            print(caractere, end = '')
            j += 1
        print('\n', end = '')
        i += 1
    return
\end{minted}
  \pause
Ce que ça donne sur \href{https://pythontutor.com/}{Python Tutor}.
\end{frame}
\begin{frame}[fragile]{Erreur fréquente : confusion \begin{tabular}{l}paramètre / saisie \\ retour / affichage\end{tabular}}

  \begin{block}{Paramètre / saisie}
\begin{minted}{python}
def maximum(a,b):
    a = int(input()) # NON !
    b = int(input()) # NON !
    if a >= b:
        return a
    else:
        return b
\end{minted}
  \end{block}
  \pause

  \begin{block}{Retour / affichage}
\begin{minted}{python}
def maximum(a,b):
    if a >= b:
        print(a) # NON !
    else:
        print(b) # NON !
\end{minted}
  \end{block}
\end{frame}

\begin{frame}
  \frametitle{Composition de fonctions}
  
  On peut appeler une fonction dans une fonction ! (et ainsi de suite)
\end{frame}

\begin{frame}[fragile]{Dessiner un carré : variante}
\begin{minted}{python}
def dessine_ligne(n, caractere):
    j = 0
    while j < n:
        print(caractere, end = '')
        j += 1
    print('\n', end = '')
    return

def dessine_carre(n, caractere):
    i = 0
    while i < n:
        dessine_ligne(n,caractere)
        i += 1
    return
\end{minted}
\end{frame}

\begin{frame}
  \frametitle{Fonctions et espaces de nom}
  \begin{itemize}
  \item les paramètres et variables définies dans le corps d'une fonction sont \alert{indépendantes} des autres variables du programme
  \item elles n'existent plus une fois l'exécution de la fonction terminée
  \item on les appelle des variables \emph{locales}
\end{itemize}

Par conséquent, on peut renommer les variables d'une fonction.

$\rightarrow{}$ Démonstration sur Thonny (\texttt{maximum.py})

\end{frame}

\begin{frame}
  \frametitle{(Contre-)exemple : intervertir des variables}
  \begin{itemize}
  \item Dans \texttt{maximum.py}, changer les valeurs de \mintinline{python}{a} et \mintinline{python}{b} dans la fonction \alert{n'a pas d'effet} sur \mintinline{python}{x} et \mintinline{python}{y} dans le programme principal !
    \item Dans \texttt{interversion.py}, la variable \mintinline{python}{temp} \alert{n'existe plus} après l'exécution de la fonction
  \end{itemize}
\end{frame}
\end{document}

%%% Local Variables:
%%% TeX-engine: xetex
%%% TeX-command-extra-options: "-shell-escape"
%%% End:
